\documentclass[12pt]{scrartcl}
\usepackage{wrapfig}
\usepackage{tabu}
\usepackage{graphicx}
	\graphicspath{{../Images/}}
\usepackage{hyperref}
	\hypersetup{
		colorlinks=true,
		linkcolor=blue,
		filecolor=magenta,      
		urlcolor=cyan,
	}

\newcommand{\n}{\newline}

\title{Potomac Curling Club Slack\\
	\large A How-To Guide}
\author{Joe Asercion}
\date{9/21/18}
\begin{document}
	\maketitle
	\section{What is slack?}
	It is easiest to think of Slack as an instant-messaging system with extra features.  Slack allows a user to:
	\begin{itemize}
		\item Send private messages to individuals or groups
		\item Create/Join/Participate in both public and private dedicated discussion channels for specific topics (i.e. Bonspiels, Thursday League, or Ice Team)
		\item Share files with both groups of people in both direct messages and public chat channels
		\item Use Slack 'Integrations' to automate things such as sending emails, making  announcements, sending reminders, finding spares, etc...
		\item 'Pin' items to discussion channels to make them easier to find by members of the channel
	\end{itemize}
	And much more!  A Slack 'workspace' is accessible via a web browser (e.g. Safari, Chrome, Firefox, etc.), the Slack desktop application, or the Slack smartphone app.
	
	\subsection{Workspace? Integration? Pinning? What on Earth are you talking about?!?}
	\textbf{Glossary of Terms: \n}
	\begin{itemize}
			\item \textbf{Workspace:} The term for an individual group's version of Slack.  As you can see here, my Slack desktop app has five different Slack workspaces, of which the PCC Slack is one.  If you don't use Slack at your place of business, it is likely that the only Slack workspace you will see is the PCC workspace.
			\item \textbf{Integration:} A piece of software which your Slack Administrator has installed into the Slack workspace to make automating different things a breeze.  Examples are an email integration, which allows users to send emails directly from Slack, a calendar integration which automatically looks up events in a Google Calendar and posts information about them to a designated location, or a reminder integration which sends users a reminder message at a time they select.  The PCC Slack has a couple of custom integrations in the form of SpareBot and CurlingTeamBot which will be discussed later in the document.
			\item \textbf{Pinning:} Slack allows users to 'pin' individual messages to a discussion channel so that they are easier to find in the future.  How to do this and how to find the pinned items will be discussed later in the document.
	\end{itemize}
	
	\section{Sounds good to me!  How do I join?}
	Simply click the link below and follow the instructions.  You can also watch this video I made which takes you through the process step by step.\n
	
	\href{https://join.slack.com/t/curldc/shared_invite/enQtNDQwODk2MDA0NTMzLTZkOWJmYjU4N2ZmMTFlMmExZmE1ZjU4NGY1ZTFkZTU1NTRjYjZkNGUyMWQ0ZjhlODZkNTU0MTg1ZTFmYTY1NDA}{Invite Link}
	
	%Insert signup video here
	
	\section{Using Slack}
	Here we get into the real meat-and-potatoes of what makes Slack great.  
	\subsection{A tour of the PCC workspace}
		\subsubsection{Accessing the workspace}
		\subsubsection{Video Tour}
	\subsection{Slack Messages}
		\subsubsection{Sending Messages}
		\textbf{Message Formatting}
		\subsubsection{Sharing Files}
		\subsubsection{Pinning Messages}
		\subsubsection{Starring Messages}
	\subsection{Discussion Channels}
		%list active discussion channels
		\subsubsection{Joining Discussion Channels}
	\subsection{Slack Integrations}
		%list installed integrations
		\subsubsection{SpareBot}
		\subsubsection{CurlingTeamBot}
	
	

\end{document}
